\PassOptionsToPackage{unicode=true}{hyperref} % options for packages loaded elsewhere
\PassOptionsToPackage{hyphens}{url}
%
\documentclass[]{article}
\usepackage{lmodern}
\usepackage{amssymb,amsmath}
\usepackage{ifxetex,ifluatex}
\usepackage{fixltx2e} % provides \textsubscript
\ifnum 0\ifxetex 1\fi\ifluatex 1\fi=0 % if pdftex
  \usepackage[T1]{fontenc}
  \usepackage[utf8]{inputenc}
  \usepackage{textcomp} % provides euro and other symbols
\else % if luatex or xelatex
  \usepackage{unicode-math}
  \defaultfontfeatures{Ligatures=TeX,Scale=MatchLowercase}
\fi
% use upquote if available, for straight quotes in verbatim environments
\IfFileExists{upquote.sty}{\usepackage{upquote}}{}
% use microtype if available
\IfFileExists{microtype.sty}{%
\usepackage[]{microtype}
\UseMicrotypeSet[protrusion]{basicmath} % disable protrusion for tt fonts
}{}
\IfFileExists{parskip.sty}{%
\usepackage{parskip}
}{% else
\setlength{\parindent}{0pt}
\setlength{\parskip}{6pt plus 2pt minus 1pt}
}
\usepackage{hyperref}
\hypersetup{
            pdftitle={Project 2},
            pdfborder={0 0 0},
            breaklinks=true}
\urlstyle{same}  % don't use monospace font for urls
\usepackage[margin=1in]{geometry}
\usepackage{color}
\usepackage{fancyvrb}
\newcommand{\VerbBar}{|}
\newcommand{\VERB}{\Verb[commandchars=\\\{\}]}
\DefineVerbatimEnvironment{Highlighting}{Verbatim}{commandchars=\\\{\}}
% Add ',fontsize=\small' for more characters per line
\usepackage{framed}
\definecolor{shadecolor}{RGB}{248,248,248}
\newenvironment{Shaded}{\begin{snugshade}}{\end{snugshade}}
\newcommand{\AlertTok}[1]{\textcolor[rgb]{0.94,0.16,0.16}{#1}}
\newcommand{\AnnotationTok}[1]{\textcolor[rgb]{0.56,0.35,0.01}{\textbf{\textit{#1}}}}
\newcommand{\AttributeTok}[1]{\textcolor[rgb]{0.77,0.63,0.00}{#1}}
\newcommand{\BaseNTok}[1]{\textcolor[rgb]{0.00,0.00,0.81}{#1}}
\newcommand{\BuiltInTok}[1]{#1}
\newcommand{\CharTok}[1]{\textcolor[rgb]{0.31,0.60,0.02}{#1}}
\newcommand{\CommentTok}[1]{\textcolor[rgb]{0.56,0.35,0.01}{\textit{#1}}}
\newcommand{\CommentVarTok}[1]{\textcolor[rgb]{0.56,0.35,0.01}{\textbf{\textit{#1}}}}
\newcommand{\ConstantTok}[1]{\textcolor[rgb]{0.00,0.00,0.00}{#1}}
\newcommand{\ControlFlowTok}[1]{\textcolor[rgb]{0.13,0.29,0.53}{\textbf{#1}}}
\newcommand{\DataTypeTok}[1]{\textcolor[rgb]{0.13,0.29,0.53}{#1}}
\newcommand{\DecValTok}[1]{\textcolor[rgb]{0.00,0.00,0.81}{#1}}
\newcommand{\DocumentationTok}[1]{\textcolor[rgb]{0.56,0.35,0.01}{\textbf{\textit{#1}}}}
\newcommand{\ErrorTok}[1]{\textcolor[rgb]{0.64,0.00,0.00}{\textbf{#1}}}
\newcommand{\ExtensionTok}[1]{#1}
\newcommand{\FloatTok}[1]{\textcolor[rgb]{0.00,0.00,0.81}{#1}}
\newcommand{\FunctionTok}[1]{\textcolor[rgb]{0.00,0.00,0.00}{#1}}
\newcommand{\ImportTok}[1]{#1}
\newcommand{\InformationTok}[1]{\textcolor[rgb]{0.56,0.35,0.01}{\textbf{\textit{#1}}}}
\newcommand{\KeywordTok}[1]{\textcolor[rgb]{0.13,0.29,0.53}{\textbf{#1}}}
\newcommand{\NormalTok}[1]{#1}
\newcommand{\OperatorTok}[1]{\textcolor[rgb]{0.81,0.36,0.00}{\textbf{#1}}}
\newcommand{\OtherTok}[1]{\textcolor[rgb]{0.56,0.35,0.01}{#1}}
\newcommand{\PreprocessorTok}[1]{\textcolor[rgb]{0.56,0.35,0.01}{\textit{#1}}}
\newcommand{\RegionMarkerTok}[1]{#1}
\newcommand{\SpecialCharTok}[1]{\textcolor[rgb]{0.00,0.00,0.00}{#1}}
\newcommand{\SpecialStringTok}[1]{\textcolor[rgb]{0.31,0.60,0.02}{#1}}
\newcommand{\StringTok}[1]{\textcolor[rgb]{0.31,0.60,0.02}{#1}}
\newcommand{\VariableTok}[1]{\textcolor[rgb]{0.00,0.00,0.00}{#1}}
\newcommand{\VerbatimStringTok}[1]{\textcolor[rgb]{0.31,0.60,0.02}{#1}}
\newcommand{\WarningTok}[1]{\textcolor[rgb]{0.56,0.35,0.01}{\textbf{\textit{#1}}}}
\usepackage{graphicx,grffile}
\makeatletter
\def\maxwidth{\ifdim\Gin@nat@width>\linewidth\linewidth\else\Gin@nat@width\fi}
\def\maxheight{\ifdim\Gin@nat@height>\textheight\textheight\else\Gin@nat@height\fi}
\makeatother
% Scale images if necessary, so that they will not overflow the page
% margins by default, and it is still possible to overwrite the defaults
% using explicit options in \includegraphics[width, height, ...]{}
\setkeys{Gin}{width=\maxwidth,height=\maxheight,keepaspectratio}
\setlength{\emergencystretch}{3em}  % prevent overfull lines
\providecommand{\tightlist}{%
  \setlength{\itemsep}{0pt}\setlength{\parskip}{0pt}}
\setcounter{secnumdepth}{0}
% Redefines (sub)paragraphs to behave more like sections
\ifx\paragraph\undefined\else
\let\oldparagraph\paragraph
\renewcommand{\paragraph}[1]{\oldparagraph{#1}\mbox{}}
\fi
\ifx\subparagraph\undefined\else
\let\oldsubparagraph\subparagraph
\renewcommand{\subparagraph}[1]{\oldsubparagraph{#1}\mbox{}}
\fi

% set default figure placement to htbp
\makeatletter
\def\fps@figure{htbp}
\makeatother


\title{Project 2}
\author{}
\date{\vspace{-2.5em}}

\begin{document}
\maketitle

\hypertarget{data-wrangling-and-exploratory-data-analysis}{%
\section{Data Wrangling and Exploratory Data
Analysis}\label{data-wrangling-and-exploratory-data-analysis}}

\hypertarget{setting-up-the-connection}{%
\subsubsection{Setting up the
connection}\label{setting-up-the-connection}}

\begin{Shaded}
\begin{Highlighting}[]
\KeywordTok{library}\NormalTok{(tidyverse)}
\end{Highlighting}
\end{Shaded}

\begin{verbatim}
## ── Attaching packages ──────────────────────────────────── tidyverse 1.3.0 ──
\end{verbatim}

\begin{verbatim}
## ✓ ggplot2 3.2.1     ✓ purrr   0.3.3
## ✓ tibble  2.1.3     ✓ dplyr   0.8.3
## ✓ tidyr   1.0.0     ✓ stringr 1.4.0
## ✓ readr   1.3.1     ✓ forcats 0.4.0
\end{verbatim}

\begin{verbatim}
## ── Conflicts ─────────────────────────────────────── tidyverse_conflicts() ──
## x dplyr::filter() masks stats::filter()
## x dplyr::lag()    masks stats::lag()
\end{verbatim}

\begin{Shaded}
\begin{Highlighting}[]
\KeywordTok{library}\NormalTok{(RSQLite)}
\NormalTok{db <-}\StringTok{ }\NormalTok{DBI}\OperatorTok{::}\KeywordTok{dbConnect}\NormalTok{(RSQLite}\OperatorTok{::}\KeywordTok{SQLite}\NormalTok{(), }\StringTok{"lahman2016.sqlite"}\NormalTok{)}
\NormalTok{DBI}\OperatorTok{::}\KeywordTok{dbListTables}\NormalTok{(db)}
\end{Highlighting}
\end{Shaded}

\begin{verbatim}
##  [1] "AllstarFull"         "Appearances"         "AwardsManagers"     
##  [4] "AwardsPlayers"       "AwardsShareManagers" "AwardsSharePlayers" 
##  [7] "Batting"             "BattingPost"         "CollegePlaying"     
## [10] "Fielding"            "FieldingOF"          "FieldingOFsplit"    
## [13] "FieldingPost"        "HallOfFame"          "HomeGames"          
## [16] "Managers"            "ManagersHalf"        "Master"             
## [19] "Parks"               "Pitching"            "PitchingPost"       
## [22] "Salaries"            "Schools"             "SeriesPost"         
## [25] "Teams"               "TeamsFranchises"     "TeamsHalf"
\end{verbatim}

\hypertarget{problem-1-using-sql-write-a-query-to-compute-the-total-payroll-and-winning-percentage-number-of-wins-number-of-games-100-for-each-team-that-is-for-each-teamid-and-yearid-combination.-you-should-include-other-columns-that-will-help-when-performing-eda-later-on-e.g.-franchise-ids-number-of-wins-number-of-games.}{%
\subsubsection{Problem 1 Using SQL, write a query to compute the total
payroll and winning percentage (number of wins / number of games * 100)
for each team (that is, for each teamID and yearID combination). You
should include other columns that will help when performing EDA later on
(e.g., franchise ids, number of wins, number of
games).}\label{problem-1-using-sql-write-a-query-to-compute-the-total-payroll-and-winning-percentage-number-of-wins-number-of-games-100-for-each-team-that-is-for-each-teamid-and-yearid-combination.-you-should-include-other-columns-that-will-help-when-performing-eda-later-on-e.g.-franchise-ids-number-of-wins-number-of-games.}}

\begin{Shaded}
\begin{Highlighting}[]
\KeywordTok{select}\NormalTok{ Salaries.yearID }\KeywordTok{as} \DataTypeTok{year}\NormalTok{, }
\NormalTok{  Teams.teamID }\KeywordTok{as}\NormalTok{ team, }
\NormalTok{  (}\FunctionTok{cast}\NormalTok{(}\FunctionTok{sum}\NormalTok{(Teams.W) }\KeywordTok{as} \DataTypeTok{float}\NormalTok{) / }\FunctionTok{sum}\NormalTok{(Teams.G))*}\DecValTok{100} \KeywordTok{as}\NormalTok{ win_percentage,}
\NormalTok{  franchName,}
  \FunctionTok{sum}\NormalTok{(Salaries.salary) }\KeywordTok{as}\NormalTok{ payroll}
\KeywordTok{from}\NormalTok{ Salaries}
\KeywordTok{join}\NormalTok{ Teams}
\KeywordTok{on}\NormalTok{ Salaries.teamID = Teams.teamID }
\KeywordTok{and}\NormalTok{ Salaries.yearID = Teams.yearID}
\KeywordTok{join}\NormalTok{ TeamsFranchises}
\KeywordTok{on}\NormalTok{ Teams.franchID = TeamsFranchises.franchID}
\KeywordTok{where} \DataTypeTok{year} \KeywordTok{between} \DecValTok{1990} \KeywordTok{and} \DecValTok{2014}
\KeywordTok{group} \KeywordTok{by} \DataTypeTok{year}\NormalTok{, team}
\end{Highlighting}
\end{Shaded}

\hypertarget{exploratory-data-analysis}{%
\section{Exploratory data analysis}\label{exploratory-data-analysis}}

\hypertarget{payroll-distribution}{%
\subsection{Payroll distribution}\label{payroll-distribution}}

\hypertarget{problem-2.-write-code-to-produce-a-plot-or-plots-that-shows-the-distribution-of-payrolls-across-teams-conditioned-on-year-from-1990-2014.-note-you-may-create-a-single-plot-as-long-as-the-distributions-for-each-year-are-clearly-distinguishable-e.g.-a-single-plot-overlaying-histograms-is-not-ok.}{%
\subsubsection{Problem 2. Write code to produce a plot (or plots) that
shows the distribution of payrolls across teams conditioned on year
(from 1990-2014). Note: you may create a single plot as long as the
distributions for each year are clearly distinguishable (e.g., a single
plot overlaying histograms is not
OK).}\label{problem-2.-write-code-to-produce-a-plot-or-plots-that-shows-the-distribution-of-payrolls-across-teams-conditioned-on-year-from-1990-2014.-note-you-may-create-a-single-plot-as-long-as-the-distributions-for-each-year-are-clearly-distinguishable-e.g.-a-single-plot-overlaying-histograms-is-not-ok.}}

\begin{Shaded}
\begin{Highlighting}[]
\KeywordTok{select}\NormalTok{ Salaries.yearID }\KeywordTok{as} \DataTypeTok{year}\NormalTok{, }
\NormalTok{  Teams.teamID }\KeywordTok{as}\NormalTok{ team, }
  \FunctionTok{sum}\NormalTok{(Salaries.salary) }\KeywordTok{as}\NormalTok{ payroll}
\KeywordTok{from}\NormalTok{ Salaries}
\KeywordTok{join}\NormalTok{ Teams}
\KeywordTok{on}\NormalTok{ Salaries.teamID = Teams.teamID }
\KeywordTok{and}\NormalTok{ Salaries.yearID = Teams.yearID}
\KeywordTok{where} \DataTypeTok{year} \KeywordTok{between} \DecValTok{1990} \KeywordTok{and} \DecValTok{2014}
\KeywordTok{group} \KeywordTok{by} \DataTypeTok{year}\NormalTok{, team}
\end{Highlighting}
\end{Shaded}

\begin{Shaded}
\begin{Highlighting}[]
\KeywordTok{library}\NormalTok{(ggplot2)}

\NormalTok{winpay }\OperatorTok\StringTok{ }
\StringTok{  }\KeywordTok{group_by}\NormalTok{(team) }\OperatorTok\StringTok{ }
\StringTok{  }\KeywordTok{mutate}\NormalTok{(}\DataTypeTok{log_payroll =} \KeywordTok{log}\NormalTok{(payroll)) }\OperatorTok\StringTok{ }
\StringTok{  }\KeywordTok{mutate}\NormalTok{(}\DataTypeTok{yearRange =} \KeywordTok{ifelse}\NormalTok{(}\DecValTok{1989} \OperatorTok{<}\StringTok{ }\NormalTok{year }\OperatorTok{&}\StringTok{ }\NormalTok{year }\OperatorTok{<=}\StringTok{ }\DecValTok{1994}\NormalTok{, }\KeywordTok{paste}\NormalTok{(}\StringTok{"1989-1994"}\NormalTok{),}
                \KeywordTok{ifelse}\NormalTok{(}\DecValTok{1994} \OperatorTok{<}\StringTok{ }\NormalTok{year }\OperatorTok{&}\StringTok{ }\NormalTok{year }\OperatorTok{<=}\StringTok{ }\DecValTok{1999}\NormalTok{, }\KeywordTok{paste}\NormalTok{(}\StringTok{"1994-1999"}\NormalTok{),}
                \KeywordTok{ifelse}\NormalTok{(}\DecValTok{1999} \OperatorTok{<}\StringTok{ }\NormalTok{year }\OperatorTok{&}\StringTok{ }\NormalTok{year }\OperatorTok{<=}\StringTok{ }\DecValTok{2004}\NormalTok{, }\KeywordTok{paste}\NormalTok{(}\StringTok{"1999-2004"}\NormalTok{),}
                \KeywordTok{ifelse}\NormalTok{(}\DecValTok{2004} \OperatorTok{<}\StringTok{ }\NormalTok{year }\OperatorTok{&}\StringTok{ }\NormalTok{year }\OperatorTok{<=}\StringTok{ }\DecValTok{2009}\NormalTok{, }\KeywordTok{paste}\NormalTok{(}\StringTok{"2004-2009"}\NormalTok{),}
                       \KeywordTok{paste}\NormalTok{(}\StringTok{"2009-2014"}\NormalTok{)))))) }\OperatorTok\StringTok{ }
\StringTok{  }\KeywordTok{ggplot}\NormalTok{(}\KeywordTok{aes}\NormalTok{(}\DataTypeTok{x=}\NormalTok{yearRange, }\DataTypeTok{y=}\NormalTok{log_payroll)) }\OperatorTok{+}\StringTok{ }
\StringTok{  }\KeywordTok{geom_boxplot}\NormalTok{() }\OperatorTok{+}
\StringTok{  }\KeywordTok{labs}\NormalTok{(}\DataTypeTok{x=}\StringTok{"Year Range"}\NormalTok{, }\DataTypeTok{y=}\StringTok{"Log Payroll"}\NormalTok{)}
\end{Highlighting}
\end{Shaded}

\includegraphics{project2_files/figure-latex/plot-1.pdf}

\hypertarget{question-1.-what-statements-can-you-make-about-the-distribution-of-payrolls-conditioned-on-time-based-on-these-plots-remember-you-can-make-statements-in-terms-of-central-tendency-spread-etc.}{%
\paragraph{Question 1. What statements can you make about the
distribution of payrolls conditioned on time based on these plots?
Remember you can make statements in terms of central tendency, spread,
etc.}\label{question-1.-what-statements-can-you-make-about-the-distribution-of-payrolls-conditioned-on-time-based-on-these-plots-remember-you-can-make-statements-in-terms-of-central-tendency-spread-etc.}}

There is a clear increasing central tendency, although it seems the
spread of the data has also slightly increased over time as well. It can
also be seen that the upper end of the payroll scale seemed to flatten
out after 2004. Most of the outliers reside on the lower end of the
payroll scale, whereas at the upper end there is only one.

\hypertarget{problem-3.-write-code-to-produce-a-plot-or-plots-that-specifically-shows-at-least-one-of-the-statements-you-made-in-question-1.-for-example-if-you-make-a-statement-that-there-is-a-trend-for-payrolls-to-decrease-over-time-make-a-plot-of-a-statistic-for-central-tendency-e.g.-mean-payroll-vs.time-to-show-that-specifically.}{%
\subsubsection{Problem 3. Write code to produce a plot (or plots) that
specifically shows at least one of the statements you made in Question
1. For example, if you make a statement that there is a trend for
payrolls to decrease over time, make a plot of a statistic for central
tendency (e.g., mean payroll) vs.~time to show that
specifically.}\label{problem-3.-write-code-to-produce-a-plot-or-plots-that-specifically-shows-at-least-one-of-the-statements-you-made-in-question-1.-for-example-if-you-make-a-statement-that-there-is-a-trend-for-payrolls-to-decrease-over-time-make-a-plot-of-a-statistic-for-central-tendency-e.g.-mean-payroll-vs.time-to-show-that-specifically.}}

The statement I am going to prove is that the maximum payrolls stopped
increasing significantly after 2004. I did this by graphing strictly the
max payrolls between the years 1990 and 2014. The graph clearly shows a
steep decline in the slope of the graph after 2005, showing that the
payrolls stopped increasing as sharply at that point.

\begin{Shaded}
\begin{Highlighting}[]
\KeywordTok{select} \DataTypeTok{year}\NormalTok{, }\FunctionTok{max}\NormalTok{(payroll) }\KeywordTok{as}\NormalTok{ max_payroll}
\KeywordTok{from}
\NormalTok{  (}\KeywordTok{select}\NormalTok{ Salaries.yearID }\KeywordTok{as} \DataTypeTok{year}\NormalTok{, }
\NormalTok{    Teams.teamID }\KeywordTok{as}\NormalTok{ team, }
    \FunctionTok{sum}\NormalTok{(Salaries.salary) }\KeywordTok{as}\NormalTok{ payroll}
  \KeywordTok{from}\NormalTok{ Salaries}
  \KeywordTok{join}\NormalTok{ Teams}
  \KeywordTok{on}\NormalTok{ Salaries.teamID = Teams.teamID}
  \KeywordTok{and}\NormalTok{ Salaries.yearID = Teams.yearID}
  \KeywordTok{where} \DataTypeTok{year} \KeywordTok{between} \DecValTok{1990} \KeywordTok{and} \DecValTok{2014}
  \KeywordTok{group} \KeywordTok{by} \DataTypeTok{year}\NormalTok{, team)}
\KeywordTok{group} \KeywordTok{by} \DataTypeTok{year}
\end{Highlighting}
\end{Shaded}

\begin{Shaded}
\begin{Highlighting}[]
\NormalTok{max_df }\OperatorTok\StringTok{ }
\StringTok{  }\KeywordTok{mutate}\NormalTok{(}\DataTypeTok{log_payroll =} \KeywordTok{log}\NormalTok{(max_payroll)) }\OperatorTok\StringTok{ }
\StringTok{  }\KeywordTok{ggplot}\NormalTok{(}\KeywordTok{aes}\NormalTok{(}\DataTypeTok{x=}\NormalTok{year, }\DataTypeTok{y=}\NormalTok{log_payroll)) }\OperatorTok{+}\StringTok{ }
\StringTok{  }\KeywordTok{geom_point}\NormalTok{()}
\end{Highlighting}
\end{Shaded}

\includegraphics{project2_files/figure-latex/max_plot-1.pdf}

\hypertarget{correlation-between-payroll-and-winning-percentage}{%
\section{Correlation between payroll and winning
percentage}\label{correlation-between-payroll-and-winning-percentage}}

\hypertarget{problem-4.-write-code-to-discretize-year-into-five-time-periods-e.g.-using-the-cut-function-with-parameter-breaks5-in-r-bins5-in-python-and-then-make-a-scatterplot-showing-mean-winning-percentage-y-axis-vs.mean-payroll-x-axis-for-each-of-the-five-time-periods.-you-could-add-a-regression-line-using-geom_smoothmethodlm-in-each-scatter-plot-to-ease-interpretation.-note-look-at-the-discussion-on-faceting-in-the-visualization-eda-lecture-notes.}{%
\subsubsection{Problem 4. Write code to discretize year into five time
periods (e.g., using the cut function with parameter breaks=5 (in R,
bins=5 in python) and then make a scatterplot showing mean winning
percentage (y-axis) vs.~mean payroll (x-axis) for each of the five time
periods. You could add a regression line (using geom\_smooth(method=lm))
in each scatter plot to ease interpretation. Note: look at the
discussion on faceting in the visualization EDA lecture
notes.}\label{problem-4.-write-code-to-discretize-year-into-five-time-periods-e.g.-using-the-cut-function-with-parameter-breaks5-in-r-bins5-in-python-and-then-make-a-scatterplot-showing-mean-winning-percentage-y-axis-vs.mean-payroll-x-axis-for-each-of-the-five-time-periods.-you-could-add-a-regression-line-using-geom_smoothmethodlm-in-each-scatter-plot-to-ease-interpretation.-note-look-at-the-discussion-on-faceting-in-the-visualization-eda-lecture-notes.}}

\begin{Shaded}
\begin{Highlighting}[]
\NormalTok{winpay}\OperatorTok{$}\NormalTok{year_range <-}\StringTok{ }\KeywordTok{cut}\NormalTok{(winpay}\OperatorTok{$}\NormalTok{year, }\DataTypeTok{breaks=}\DecValTok{5}\NormalTok{)}
\NormalTok{winpay}\OperatorTok{$}\NormalTok{Franchise[winpay}\OperatorTok{$}\NormalTok{franchName }\OperatorTok{!=}\StringTok{ 'Oakland Athletics'}\NormalTok{] <-}\StringTok{ 'Other'}
\NormalTok{winpay}\OperatorTok{$}\NormalTok{Franchise[winpay}\OperatorTok{$}\NormalTok{franchName }\OperatorTok{==}\StringTok{ 'Oakland Athletics'}\NormalTok{] <-}\StringTok{ 'Oakland'}
\NormalTok{winpay }\OperatorTok\StringTok{ }
\StringTok{  }\KeywordTok{mutate}\NormalTok{(}\DataTypeTok{log_payroll =} \KeywordTok{log}\NormalTok{(payroll)) }\OperatorTok
\StringTok{  }\KeywordTok{ggplot}\NormalTok{(}\KeywordTok{aes}\NormalTok{(}\DataTypeTok{x=}\NormalTok{log_payroll, }\DataTypeTok{y=}\NormalTok{win_percentage, }\DataTypeTok{color=}\NormalTok{Franchise)) }\OperatorTok{+}\StringTok{ }
\StringTok{  }\KeywordTok{labs}\NormalTok{(}\DataTypeTok{title=}\StringTok{'Win Percentage vs Payroll'}\NormalTok{) }\OperatorTok{+}
\StringTok{  }\KeywordTok{facet_grid}\NormalTok{(}\OperatorTok{~}\NormalTok{year_range) }\OperatorTok{+}
\StringTok{  }\KeywordTok{geom_point}\NormalTok{() }\OperatorTok{+}
\StringTok{  }\KeywordTok{geom_smooth}\NormalTok{(}\DataTypeTok{method=}\NormalTok{lm, }\DataTypeTok{color=}\StringTok{'blue'}\NormalTok{)}
\end{Highlighting}
\end{Shaded}

\includegraphics{project2_files/figure-latex/time_periods-1.pdf}

\hypertarget{question-2.-what-can-you-say-about-team-payrolls-across-these-periods-are-there-any-teams-that-standout-as-being-particularly-good-at-paying-for-wins-across-these-time-periods-what-can-you-say-about-the-oakland-as-spending-efficiency-across-these-time-periods-labeling-some-points-in-the-scatterplot-can-help-interpretation.}{%
\paragraph{Question 2. What can you say about team payrolls across these
periods? Are there any teams that standout as being particularly good at
paying for wins across these time periods? What can you say about the
Oakland A's spending efficiency across these time periods (labeling some
points in the scatterplot can help
interpretation).}\label{question-2.-what-can-you-say-about-team-payrolls-across-these-periods-are-there-any-teams-that-standout-as-being-particularly-good-at-paying-for-wins-across-these-time-periods-what-can-you-say-about-the-oakland-as-spending-efficiency-across-these-time-periods-labeling-some-points-in-the-scatterplot-can-help-interpretation.}}

I can say that as a general trend, win percentages increase as a team's
payroll increases, but there's so much variance in these plots that it
is hard to call this a strong causation relationship. It seems that the
time period when these two variables were most strongly correlated was
the 1995 to 2000 time period. The Oakland A's spending efficiency seems
to be solid. Acroess the five time periods, they generally end to be
either above the mean, or very close to it, except for the 1990-1995
time range.

\hypertarget{data-transformations}{%
\section{Data transformations}\label{data-transformations}}

\hypertarget{standardization-across-years}{%
\subsection{Standardization across
years}\label{standardization-across-years}}

\hypertarget{problem-5.-write-code-to-create-a-new-variable-in-your-dataset-that-standardizes-payroll-conditioned-on-year.}{%
\subsubsection{Problem 5. Write code to create a new variable in your
dataset that standardizes payroll conditioned on
year.}\label{problem-5.-write-code-to-create-a-new-variable-in-your-dataset-that-standardizes-payroll-conditioned-on-year.}}

\begin{Shaded}
\begin{Highlighting}[]
\NormalTok{payroll_mew =}\StringTok{ }\KeywordTok{array}\NormalTok{()}
\NormalTok{payroll_sigma =}\StringTok{ }\KeywordTok{array}\NormalTok{()}
\ControlFlowTok{for}\NormalTok{(i }\ControlFlowTok{in} \DecValTok{1990}\OperatorTok{:}\DecValTok{2014}\NormalTok{) \{}
\NormalTok{  payroll_mew[i }\OperatorTok{-}\StringTok{ }\DecValTok{1990} \OperatorTok{+}\StringTok{ }\DecValTok{1}\NormalTok{] <-}\StringTok{ }\KeywordTok{mean}\NormalTok{(}\KeywordTok{filter}\NormalTok{(winpay, year}\OperatorTok{==}\NormalTok{i)}\OperatorTok{$}\NormalTok{payroll)}
\NormalTok{  payroll_sigma[i }\OperatorTok{-}\StringTok{ }\DecValTok{1990} \OperatorTok{+}\StringTok{ }\DecValTok{1}\NormalTok{] <-}\StringTok{ }\KeywordTok{sd}\NormalTok{(}\KeywordTok{filter}\NormalTok{(winpay, year}\OperatorTok{==}\NormalTok{i)}\OperatorTok{$}\NormalTok{payroll)}
\NormalTok{\}}


\NormalTok{winpay <-}\StringTok{ }\NormalTok{winpay }\OperatorTok\StringTok{ }
\StringTok{  }\KeywordTok{mutate}\NormalTok{(}\DataTypeTok{Payroll_Z_Score =} 
\NormalTok{(payroll }\OperatorTok{-}\StringTok{ }\NormalTok{payroll_mew[year }\OperatorTok{-}\StringTok{ }\DecValTok{1990} \OperatorTok{+}\StringTok{ }\DecValTok{1}\NormalTok{]) }\OperatorTok{/}\StringTok{ }\NormalTok{payroll_sigma[year }\OperatorTok{-}\StringTok{ }\DecValTok{1990} \OperatorTok{+}\StringTok{ }\DecValTok{1}\NormalTok{])}
\end{Highlighting}
\end{Shaded}

\hypertarget{problem-6.-repeat-the-same-plots-as-problem-4-but-use-this-new-standardized-payroll-variable.}{%
\subsubsection{Problem 6. Repeat the same plots as Problem 4, but use
this new standardized payroll
variable.}\label{problem-6.-repeat-the-same-plots-as-problem-4-but-use-this-new-standardized-payroll-variable.}}

\begin{Shaded}
\begin{Highlighting}[]
\NormalTok{winpay }\OperatorTok\StringTok{ }
\StringTok{  }\KeywordTok{ggplot}\NormalTok{(}\KeywordTok{aes}\NormalTok{(}\DataTypeTok{x=}\NormalTok{Payroll_Z_Score, }\DataTypeTok{y=}\NormalTok{win_percentage, }\DataTypeTok{color=}\NormalTok{Franchise)) }\OperatorTok{+}\StringTok{ }
\StringTok{  }\KeywordTok{labs}\NormalTok{(}\DataTypeTok{title=}\StringTok{'Win Percentage vs Payroll Z-Score'}\NormalTok{) }\OperatorTok{+}
\StringTok{  }\KeywordTok{facet_grid}\NormalTok{(}\OperatorTok{~}\NormalTok{year_range) }\OperatorTok{+}
\StringTok{  }\KeywordTok{geom_point}\NormalTok{() }\OperatorTok{+}
\StringTok{  }\KeywordTok{geom_smooth}\NormalTok{(}\DataTypeTok{method=}\NormalTok{lm, }\DataTypeTok{color=}\StringTok{'blue'}\NormalTok{)}
\end{Highlighting}
\end{Shaded}

\includegraphics{project2_files/figure-latex/standardized_plot-1.pdf}

\hypertarget{question-3.-discuss-how-the-plots-from-problem-4-and-problem-6-reflect-the-transformation-you-did-on-the-payroll-variable.-consider-data-range-center-and-spread-along-with-observed-correlation-in-your-discussion.-some-of-these-change-after-the-transformation-others-dont.}{%
\paragraph{Question 3. Discuss how the plots from Problem 4 and Problem
6 reflect the transformation you did on the payroll variable. Consider
data range, center and spread along with observed correlation in your
discussion. Some of these change after the transformation, others
don't.}\label{question-3.-discuss-how-the-plots-from-problem-4-and-problem-6-reflect-the-transformation-you-did-on-the-payroll-variable.-consider-data-range-center-and-spread-along-with-observed-correlation-in-your-discussion.-some-of-these-change-after-the-transformation-others-dont.}}

The most obvious difference is the slope of the trendline has noticably
increased, albeit not dramatically. The variance of the data has
increased slightly, contributing to the increase in the trendline, but
is still nearly the same. Each graph is also centered around x=0,
whereas the graphs in problem 4 each had their own center-point making
them harder to compare. Also each graph shows that as x=0, the win
percentage is consistently 50\%, implying when a team has the average
payroll amount, they win 50\% of the time.

\hypertarget{expected-wins}{%
\section{Expected wins}\label{expected-wins}}

\hypertarget{problem-7.-make-a-single-scatter-plot-of-winning-percentage-y-axis-vs.standardized-payroll-x-axis.-add-a-regression-line-to-highlight-the-relationship-again-using-geom_smoothmethodlm.}{%
\subsubsection{Problem 7. Make a single scatter plot of winning
percentage (y-axis) vs.~standardized payroll (x-axis). Add a regression
line to highlight the relationship (again using
geom\_smooth(method=lm)).}\label{problem-7.-make-a-single-scatter-plot-of-winning-percentage-y-axis-vs.standardized-payroll-x-axis.-add-a-regression-line-to-highlight-the-relationship-again-using-geom_smoothmethodlm.}}

\begin{Shaded}
\begin{Highlighting}[]
\NormalTok{winpay }\OperatorTok\StringTok{ }
\StringTok{  }\KeywordTok{ggplot}\NormalTok{(}\KeywordTok{aes}\NormalTok{(}\DataTypeTok{x=}\NormalTok{Payroll_Z_Score, }\DataTypeTok{y=}\NormalTok{win_percentage)) }\OperatorTok{+}
\StringTok{  }\KeywordTok{geom_point}\NormalTok{() }\OperatorTok{+}
\StringTok{  }\KeywordTok{geom_smooth}\NormalTok{(}\DataTypeTok{method=}\NormalTok{lm)}
\end{Highlighting}
\end{Shaded}

\includegraphics{project2_files/figure-latex/plot_all-1.pdf}

\hypertarget{spending-efficiency}{%
\section{Spending efficiency}\label{spending-efficiency}}

\hypertarget{problem-8.-write-code-to-calculate-spending-efficiency-for-each-team.-make-a-line-plot-with-year-on-the-x-axis-and-efficiency-on-the-y-axis.-a-good-set-of-teams-to-plot-are-oakland-the-new-york-yankees-boston-atlanta-and-tampa-bay-teamids-oak-bos-nya-atl-tba.-that-plot-can-be-hard-to-read-since-there-is-so-much-year-to-year-variation-for-each-team.-one-way-to-improve-it-is-to-use-geom_smooth-instead-of-geom_line.}{%
\subsubsection{Problem 8. Write code to calculate spending efficiency
for each team. Make a line plot with year on the x-axis and efficiency
on the y-axis. A good set of teams to plot are Oakland, the New York
Yankees, Boston, Atlanta and Tampa Bay (teamIDs OAK, BOS, NYA, ATL,
TBA). That plot can be hard to read since there is so much year to year
variation for each team. One way to improve it is to use geom\_smooth
instead of
geom\_line.}\label{problem-8.-write-code-to-calculate-spending-efficiency-for-each-team.-make-a-line-plot-with-year-on-the-x-axis-and-efficiency-on-the-y-axis.-a-good-set-of-teams-to-plot-are-oakland-the-new-york-yankees-boston-atlanta-and-tampa-bay-teamids-oak-bos-nya-atl-tba.-that-plot-can-be-hard-to-read-since-there-is-so-much-year-to-year-variation-for-each-team.-one-way-to-improve-it-is-to-use-geom_smooth-instead-of-geom_line.}}

\begin{Shaded}
\begin{Highlighting}[]
\NormalTok{winpay <-}\StringTok{ }\NormalTok{winpay }\OperatorTok\StringTok{ }
\StringTok{  }\KeywordTok{mutate}\NormalTok{(}\DataTypeTok{expected_win_pct =} \DecValTok{50} \OperatorTok{+}\StringTok{ }\FloatTok{2.5} \OperatorTok{*}\StringTok{ }\NormalTok{Payroll_Z_Score) }\OperatorTok\StringTok{ }
\StringTok{  }\KeywordTok{mutate}\NormalTok{(}\DataTypeTok{efficiency =}\NormalTok{ win_percentage }\OperatorTok{-}\StringTok{ }\NormalTok{expected_win_pct) }\OperatorTok\StringTok{ }
\StringTok{  }\KeywordTok{mutate}\NormalTok{(}\DataTypeTok{team =} 
           \KeywordTok{ifelse}\NormalTok{(}
\NormalTok{             team }\OperatorTok\StringTok{ }\KeywordTok{c}\NormalTok{(}\StringTok{'OAK'}\NormalTok{, }\StringTok{'BOS'}\NormalTok{, }\StringTok{'NYA'}\NormalTok{, }\StringTok{'ATL'}\NormalTok{, }\StringTok{'TBA'}\NormalTok{), }
\NormalTok{             team, }\StringTok{'OTHER'}\NormalTok{))}

\NormalTok{winpay }\OperatorTok\StringTok{ }
\StringTok{  }\KeywordTok{ggplot}\NormalTok{(}\KeywordTok{aes}\NormalTok{(}\DataTypeTok{x=}\NormalTok{year, }\DataTypeTok{y=}\NormalTok{efficiency, }\DataTypeTok{color=}\NormalTok{team)) }\OperatorTok{+}
\StringTok{  }\KeywordTok{geom_smooth}\NormalTok{(}\DataTypeTok{method=}\NormalTok{loess)}
\end{Highlighting}
\end{Shaded}

\includegraphics{project2_files/figure-latex/effic-1.pdf}

\hypertarget{question-4.-what-can-you-learn-from-this-plot-compared-to-the-set-of-plots-you-looked-at-in-question-2-and-3-how-good-was-oaklands-efficiency-during-the-moneyball-period}{%
\paragraph{Question 4. What can you learn from this plot compared to the
set of plots you looked at in Question 2 and 3? How good was Oakland's
efficiency during the Moneyball
period?}\label{question-4.-what-can-you-learn-from-this-plot-compared-to-the-set-of-plots-you-looked-at-in-question-2-and-3-how-good-was-oaklands-efficiency-during-the-moneyball-period}}

This plot provides significantly more valuable information than the
plots from questions 2 and 3. This plot shows which teams are worth
analyzing based on their efficiency throughout time. For example, if I
were going to start a baseball team, I would analyze Oakland's
strategies for team building in 2002, where they had an efficiency of
16.3\% despite consistently being the lower end of the payroll spectrum.
Oakland's efficiency during the Moneyball period was excellent. As this
chart shows, Oakland started off with about average efficiency, then
gradually throughout the 1990s their win percentage went up and up until
it peaked in 2002. During the Moneyball period Oakland's win percentage
was substantially above average and their win percentage stayed above
average at least from roughly 1997 to 2015.

\end{document}
